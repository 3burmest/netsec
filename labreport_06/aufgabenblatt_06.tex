\documentclass[ngerman]{fbi-aufgabenblatt}
\usepackage{enumitem} %Anpassbare Enumerates/Itemizes
\usepackage{biblatex}
\usepackage{listings}
\usepackage{color}
\usepackage[latin1]{inputenc}
\usepackage[T1]{fontenc}
\definecolor{pblue}{rgb}{0.13,0.13,1}
\definecolor{pgreen}{rgb}{0,0.5,0}
\definecolor{pred}{rgb}{0.9,0,0}
\definecolor{pgrey}{rgb}{0.46,0.45,0.48}

\usepackage{listings}
\lstset{language=Java,
  showspaces=false,
  showtabs=false,
  breaklines=true,
  showstringspaces=false,
  breakatwhitespace=true,
  commentstyle=\color{pgreen},
  keywordstyle=\color{pblue},
  stringstyle=\color{pred},
  basicstyle=\ttfamily,
  moredelim=[il][\textcolor{pgrey}],
  moredelim=[is][\textcolor{pgrey}]{\%\%}{\%\%}
  }

% Folgende Angaben bitte anpassen

\renewcommand{\Vorlesung}{Bachelor-Projekt Network-Security}
\renewcommand{\Semester}{SoSe 2016}

\renewcommand{\Aufgabenblatt}{6}
\renewcommand{\Teilnehmer}{Mader, Burmester}

\begin{document}

\section*{Kryptographie}

\section{Absicherung des TCP-Chats mit SSL}

\begin{enumerate}[label=\arabic*., start=1]
\item
\end{enumerate}

\section{CAs und Webserver-Zertifikate}

\begin{enumerate}[label=\arabic*., start=1]
\item
\end{enumerate}

\section{Unsichere selbstentwickelte Verschl�sselungsalgorithmen}

\begin{enumerate}[label=\arabic*., start=1]
\item
\end{enumerate}

\section{EasyEAS}

\begin{enumerate}[label=\arabic*., start=1]
\item
\end{enumerate}

\section{Timing-Angriff auf Passw�rter (Bonusaufgabe)}

\begin{enumerate}[label=\arabic*., start=1]
\item
\end{enumerate}

\newpage
\appendix
\begin{appendix}
\addcontentsline{toc}{chapter}{Anhang} 
\section{Anhang}

\end{appendix}
\end{document}
